% compile with xelatex {filename}.tex

\documentclass{article}
\usepackage{fontspec}
\usepackage{polyglossia}
\usepackage{amsmath}
\usepackage{amssymb}
\setmainlanguage{russian} 
\setotherlanguage{english}

\newfontfamily\russianfont[Script=Cyrillic]{Linux Libertine O}

\begin{document}

  \bgroup
  \def\arraystretch{1.5}%
    \begin{tabular}{ l l l l }
      \hline
      \textbf{Типы} & \textbf{Логика} & \textbf{Множества} & \textbf{Гомотопия} \\
      \hline
      \(A\) & гипотеза & множество & пространство \\
      \(a : A\) & доказательство & элемент & точка \\
      \(B(x)\) & предикат & семейство элементов & расслоение \\
      \(b(x) : B(x)\) & условное доказательство & семейство элементов & секция \\
      \(\textbf{0, 1}\) & \(\bot, \top\) & \(\varnothing, \{\varnothing\}\) & \(\varnothing, *\) \\
      \(A + B\) & \(A \vee B\) & несвязное объединение & коумножение \\
      \(A \times B\) & \(A \wedge B\) & множество пар & произведение пространств \\
      \(A \to B\) & \(A \Rightarrow B\) & множество функций & пространство функций \\
      \(\sum_{(x:A)} B(x)\) & \(\exists_{x:A} B(x)\) & несвязная сумма & общее пространство \\
      \(\prod_{(x:A)} B(x)\) & \(\forall_{x:A} B(x)\) & умножение & пространство секций \\
      \(Id_A\) & равенство = & \(\{ (x,x) \ | \ x \in A \}\) & пространство путей \(A^I\) \\
    \end{tabular}
  \egroup

\end{document}
